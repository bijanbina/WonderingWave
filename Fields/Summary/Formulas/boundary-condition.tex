If $H_2$ is inside perfect conductor then 
\setlength{\abovedisplayshortskip}{7pt}
\setlength{\belowdisplayshortskip}{7pt}
\setlength{\abovedisplayskip}{7pt}
\setlength{\belowdisplayskip}{7pt}
$$\hat{n} \times ( H_2 - H_1 ) = J_s ~~ , ~~ \hat{n} \times ( E_2 - E_1 ) = \frac{\partial B}{\partial t} ~~ , ~~ \hat{n} \cdot ( B_2 - B_1 ) = 0 ~~ , ~~ \hat{n} \cdot ( D_2 - D_1 ) = \rho $$
If $H_2$ is inside perfect conductor then 
$$ \hat{n} \times H_1 = J_s $$
where $\hat{n}$ is toward \cite{inan} Figure 1.2b. For non perfect conductor as a way in shown in pozar for calculation of power we must assume a countor to infinity and integrate it to where $E$ become zero then we gain an equation similar to $ \hat{n} \times H_1 = J_s $.

\subsection{A Note on H Boundry Condition}
The $J$ in the boundry condition must be orthogonal to the loop but this is barely considered in the literature and the reader may wonder if the solution is get wrongs. To solve this the loop must be orthogonal to the current or if the current extracted, the loop can be calculated on all directions (x, y, z)
