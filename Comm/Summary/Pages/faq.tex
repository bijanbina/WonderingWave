%This file contains frequently asked questions
\textbf{Q:} What is power spectrum density (PSD) and what is it's relation with signal spectrum?\\
\textbf{A:} PSD is the same as fourier transform of signal. This is not simply to shown but known as wiener-khinchin theorem which states that fourier transform of auto-correlation function is an indirect way to calculate spectrum density of a signal. In a deterministic signals this method is vain but in a probabilistic signal this is a valuable method as the direct method can't be used\\
\\\textbf{Q:} What is group velocity?\\
\textbf{A:} Phase velocity which is calculate from\\
$$ v = \frac{\omega}{\beta} $$
is the speed of single frequency propagate down to space. In many cases transmission signal is consists of more than a single frequency which is the velocity for each single wave is slightly different due to phase velocity frequency dependent nature. In this case we define group velocity, the speed which the wave packet ( overall shape of the waves ) propagate, please understand that group velocity is just an approximated value to the wave packet speed. and it's only valid for $\Delta \omega \ll \omega_{0}$.\\
$$ v_g = \frac{d\omega}{d\beta} $$
\\\textbf{Q:} Why we put inductance and capacitance into transmission line?\\
\textbf{A:} We put them because they exist in the line. They are exist in all frequencies but they have become considerable only in high frequencies so we have to calculate each wire inductance and the coupling factor between lines. If we deal with distributed circuits two phenomenon occurred at the same time, propagation delay and transmission line intrinsic Impedance. note that these two events are developed independently of each other to model individual wave propagation aspect.
For a mid range high frequency the PCB tracks are only exhibit as a pure resistance aka lossless line but if the frequency move toward microwave frequencies, skin-depth effect, in the practical design must take into accounts
\\\\\textbf{Q:} What is $Z_{in}$ and what is it difference with $Z_o$\\
\textbf{A:} $Z_{in}$ only defined where the line is in the steady-state. on the other hand $Z_o$ exist every where on the transmission line. It relates $V^+$ to $I^+$ independently of the load and generator. $Z_o$ employed in calculating $V^+$ when a step pulse applied to the transmission line with\\
$$ V^+ = \frac{Z_o}{Z_o + Z_g} V_g $$
where $Z_g$ is generator output resistance
on the other hand $Z_in$ relates the $V(z)$ to $I(z)$ in any point on the transmission line but only for the steady state\\
$$ Z_{in} = \frac{V_s}{I_s} $$
