%\thefontsize 
~\vspace{-2.5em}
\newcommand{\FourierTransform}{~ \xrightarrow{ \text{\large $~\mathscr{F}~$}} ~}
\subsection{Integration}
Integration is inverse of differentiation in many respect. Using this if average $x(t)$ is 0 then

\begin{equation}
\int x(t) dt \FourierTransform \frac{1}{j \omega t} X(f)
\end{equation}

To find DC value of this use average on $ \int x(t) dt $

the Real value is 
\begin{equation}
\int x(t) dt \FourierTransform \frac{1}{j \omega t} X(f) + \frac{1}{2}X(0)\delta(f)
\end{equation}

the additive value is add to a function which it's average is zero. for example for square wave this is a square wave which oscilate between $\frac{1}{2}$ and $-\frac{1}{2}$

\subsection{Fourier Series}
To find Fourier series from Fourier transform

\begin{equation}
c_n = \frac{1}{T} X(n f_0)
\end{equation}

\subsection{Important Fourier Transformation}
To find Fourier series from Fourier transform
$$\Pi (t) = 
\left\{
	\begin{array}{ll}
		1  & \mbox{ } |t|<\frac{1}{2} \\
		0 & \mbox{ } o.w
	\end{array}
\right.
$$
$$
\renewcommand{\arraystretch}{3}
	\begin{array}{cc}
		~~~~~~~~\Pi(f)~~~~~~~~   & ~~~~~~~~\dfrac{sin(\pi t)}{\pi t}~~~~~~~~ \\
		\text{LP~with}~\omega_c & \dfrac{sin(\omega_c t)}{\pi t}
	\end{array}
$$

~\\
